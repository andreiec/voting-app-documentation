\chapter{Concluzii}

În final, Aplicația Voting App a fost gândită cu scopul de a facilita un mediu cât mai eficient și mai sigur pentru exprimarea voturilor în cadrul Consiliului Facultății de Matematică și Informatică. Aceasta a apărut ca urmare a pandemiei de Covid-19, fapt ce a îngreunat toate activitățile ce se desfășurau într-un format fizic. Aplicația combină o suită de tehnologii variate, atât pe partea de client, cât și pe cea de server, aducând, pe deasupra, un design minimalist și ușor de interacționat. Dezvoltarea acestui proiect a reprezentat o posibilitate foarte bună de a înțelege și a aprofunda și mai mult cunoștințe în domeniul aplicațiilor web și al modului de utilizare a tehnologiilor implementate.

Desigur, există loc pentru îmbunătățiri și pentru implementat noi funcționalități. Printre acestea numărându-se: crearea în masă a conturilor de utilizator folosind documente de tip \enquote{.csv}, posibilitatea de a oferi mandat altui membru din consiliu pentru a vota în locul acestuia și nu numai. De asemenea, o altă versiune a acestei aplicații ar putea fi dezvoltată pentru a eficientiza procesul de alegere al studenților reprezentanți, dat fiind funcționalitatea de împărțire în grupuri pe care Voting App o are implementată.