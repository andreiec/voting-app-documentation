\section*{Rezumat}

Încă din cele mai vechi timpuri, democrația a rămas unul din conceptele cele mai vitale în majoritatea civilizațiilor. Aceasta a ajutat să transforme lumea din jurul oamenilor, dându-le acces la unul dintre cele mai puternice instrumente - votul. În acest sens, oamenii au inventat sisteme din ce în ce mai complexe pentru a gestiona numărul exponențial de oameni chemați să voteze și, mai nou, pentru a rezolva problema unei incertitudini epidemiologice. Astfel, prezenta lucrare de licență vizează implementarea unui de sistem de vot digital pentru Consiliul Facultății.

Scopul aplicației este acela de a oferi membrilor consiliului o modalitate de a-și exprima votul și prin intermediul online-ului, folosind diverse tehnologii moderne de realizare a aplicațiilor web. Printre aceste tehnologii se numără: React, utilizat în dezvoltarea interfeței, Django, folosit în partea de procesare a apelurilor făcute de către interfață și MySQL, pentru gestionarea bazei de date. De asemenea, alături de tehnologiile menționate, au fost folosite librării precum Redux, Chakra UI, React Router și Axios, toate utilizate cu scopul de a eficientiza și a ajuta la o securitate cât mai bună. Rezultatul final este o aplicație ce permite realizarea de procese electorale digitale, astfel încât, în cazul unei noi situații epidemiologice nefavorabile, Consiliul Facultății să își poată continua activitatea cât mai eficace.

\newpage

\section*{Abstract}

Democracy has been one of the most lasting ways in which civilizations have been organized through history. It helped transform the world, as the leadership was centered around the people by giving them access to one of the most powerful tools - the vote. As a consequence, one has invented increasingly complex voting systems capable to manage the exponential increase in the number of people called to vote and, more recently, to solve the problem of epidemiological uncertainty. Thus the objective of the current bachelor's thesis is to develop and implement a digital voting system for the Faculty Council.

The application aims to provide the Faculty’s Board members with a way to express their vote online, using a variety of modern web application technologies, among which: React, used in the development of the interface, Django, used in the part processing of calls made by the interface and MySQL, for database management. Along the above-mentioned technologies, there were used libraries such as Redux, Chakra UI, React Router and Axios, in order to increase effectiveness and security. The end result is an application that allows the realization of digital electoral processes, so that, in case of a new pandemic, the Faculty Council can continue its activity as effectively as possible.