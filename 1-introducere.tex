\section*{Introducere}

Trăim în era în care tehnologia a devenit parte din rutina noastră zilnică și, conform unui studiu afiliat cu Universitatea din California \cite{online_vs_offline_communication}, am ajuns să comunicăm mai mult în online decât față în față. Acest fapt a jucat un rol important în timpul pandemiei de Covid-19, când, datorită faptului că populația generală a fost supusă unor restricții ce limita, temporar, interacțiunea socială, oamenii au reușit să se adapteze noului stil de viață și, de asemenea, comunicării în online. Multe dintre activitățile ce se desfășurau față în față au fost nevoite să fie trecute în format digital. De la o simplă oră de curs în cadrul facultății, până la un job full-time, oamenii au fost nevoiți să găsească mijloace pentru a putea continua să își desfășoare aceste activități.

Printre entitățile afectate de pandemia de Covid-19, se numără, în mare parte, și cele care erau nevoite să își consolideze direcțiile decizionale printr-un vot al membrilor acestora. Ele au fost constrânse să își mute ședințele într-un format digital și, implicit, toate procesele electorale realizate. Acest fapt a adus o schimbare în modul în care erau stabilite hotărârile, de la modul clasic - fizic, la un simplu chestionar, în online. În acest fapt, aplicația vizează implementarea unei platforme digitale de vot ce are ca scop principal eficientizarea realizării proceselor electorale în cadrul Consiliului Facultății.

În următoarele capitole se va prezenta, atât soluția propusă pentru o astfel de platformă, cât și tehnologiile folosite și procesul de implementarea a acestora. Primul capitol are un rol introductiv, în care vor fi detaliate concepte din arhitectura aplicației. Al doilea capitol va prezentarea implementarea conceptelor teoretice, iar ultimul capitol se va referi la utilizarea aplicației, atât din punctul de vedere al unui utilizator, cât și al unui administrator.
